\section{Corrections for the mismatch between true and empirically derived $\pi_1$, effect size and power \label{App.corrections}}

We first introduce some notation in Table \ref{notation}:\\

\renewcommand{\arraystretch}{0.8}% Tighter
\begin{table}[H]
  \begin{center}
    \begin{tabular}{ll}
      \toprule
    $J$ & Total number of peaks \\
    \midrule
    ${\cal J}_0^u$ & Indices of peaks arising in null regions above $u$ \\
    ${\cal J}_1^u$ & Indices of peaks arising in non-null regions above $u$ \\
    $\tilde{\cal J}_0^u$ & Indices of peaks arising in empirically derived null regions above $u$ \\
    $\tilde{\cal J}_1^u$ & Indices of peaks arising in empirically derived non-null regions above $u$ \\
     & $J_0^u = |{\cal J}_0^u|$, $J_1^u = |{\cal J}_1^u|$\\
     & $\tilde{J}_0^u = |\tilde{\cal J}_0^u|$, $\tilde{J}_1^u = |\tilde{\cal J}_1^u|$\\
     \midrule
    $\pi_{00}^u$ & Proportion of $\tilde{\cal J}_0^u$ that is truly null \\
    $\pi_{10}^u$ & Proportion of $\tilde{\cal J}_1^u$ that is truly null \\
    \bottomrule
    \end{tabular}
    \caption{Notation for correction of population level estimators of $\pi_0$ and $\mu_1$. \label{notation}}
\end{center}
\end{table}

\subsection{Correction of model estimates}

As our reference level is analysed with family-wise error rate (FWER) control, we expect $\pi_{10}^u$ to be negligable and we set it to 0. $\pi_{00}^u$ can be estimated using the Beta-Uniform Model by \citet{Pounds2003}.  With these definitions, we can derive the number of peaks that are falsely classified in the FWER-analysis for the \textbf{Held-in pilot data} in table \ref{classFWE}:\\

\begin{table}[H]
  \begin{center}
\begin{tabular}{l|cc|l}
  \toprule
  & Empirically derived & Empirically derived \\
  & Null peaks & Active peaks \\
  \midrule
True null peaks & $\pi_{00}^u\tilde{J}_0^u$ & $\pi_{10}^u\tilde{J}_1^u =0$& $J_0^u$ \\
  True active peaks & $(1-\pi_{00}^u)\tilde{J}_0^u$ & $(1-\pi_{10}^u)\tilde{J}_1^u = \tilde{J}_1^u$  & $J_1^u$ \\
  \midrule
  & $\tilde{J}_0^u$ & $\tilde{J}_1^u$ & \\
  \bottomrule
\end{tabular}
\caption{Classification table of peaks after FWE thresholding in the held-in pilot data (with thresholding at $u$). \label{classFWE}}
\end{center}
\end{table}

Thus, an uncontaminated estimate of $\pi_0$ can be found as:


\begin{equation}
\tilde{\pi}_0 = \frac{J_0^u}{J^u} = \frac{\hat\pi_{00}^u\tilde{J}_0^u}{J^u} \nonumber
\end{equation}


Similarly, bias-corrected versions of $\mu_1$ are possible.  First we note

\begin{align}
E(Z^u | {\cal J}_0^u) &= u + 1/u \nonumber \\
E(Z^u | {\cal J}_1^u) &= \mu_1 \nonumber
\end{align}

where the conditional expectation indicates the set of peaks under consideration.  Now, similar to table \ref{classFWE}, we can decompose the expectation over observable sets:

\begin{align}
  E(Z^u|\tilde{\cal J}_0^u) &= \pi_{00}^u(u+1/u) + (1-\pi_{00}^u)E(Z^u | {\cal J}_1^u) \nonumber \\
  E(Z^u|\tilde{\cal J}_1^u) &= E(Z^u | {\cal J}_1^u) \nonumber
\end{align}

Thus $\tilde\mu_1$ can be estimated:

\begin{equation}
  \tilde\mu_1 = \frac{(1-\pi_{00}^u)\tilde{J}_0^u}{J_1^u}\hat E(Z^u | \tilde{J}_1^u) + \frac{\tilde{J}_1^u}{J^u_1}\frac{E(Z^u | \tilde{\cal J}_0^u)-\hat\pi_{00}^u(u+1/u)}{1-\hat\pi_{00}^u}
\end{equation}

\subsection{Correction of power estimates}

{\color{Cyan}For that for the power estimation procedure, we aim to estimate power without threshold $u$.  Therefore, we use the same notation as in \ref{notation}, but drop the $u$ to represent the peak indices without applying a threshold.  Similar to \ref{classFWE}, we can derive the number of peaks that are falsely classified in the FWER-analysis for the \textbf{Held-in study data} in table \ref{classFWEnothres}:\\

\begin{table}[H]
  \begin{center}
\begin{tabular}{l|cc|l}
  \toprule
  & Empirically derived & Empirically derived \\
  & Null peaks & Active peaks \\
  \midrule
True null peaks & $\pi_{00}^u\tilde{J}_0$ & $0$& $J_0$ \\
  True active peaks & $(1-\pi_{00})\tilde{J}_0$ & $\tilde{J}_1$  & $J_1$ \\
  \midrule
  & $\tilde{J}_0$ & $\tilde{J}_1$ & \\
  \bottomrule
\end{tabular}
\caption{Classification table of peaks after FWE thresholding in the held-in study data (without thresholding). \label{classFWEnothres}}
\end{center}
\end{table}

We again estimate $\pi_00$, the proportion of active peaks among all empirically derived null peaks, $\tilde{J}_0$, using the Beta-Uniform Model.  Using table \ref{classFWEnothres}, an uncontaminated estimate of $1-\beta_{z_\alpha}$ can be found as:

\begin{align}
    (1-\tilde\beta_{z_\alpha}) &= \frac{|Z_j \geq z_\alpha|}{J_1}, \text{ for } j \in \tilde{\mathcal{J}_1} \nonumber \\
                              &= \frac{|Z_j \geq z_\alpha|}{\tilde{J}_1 + (1-\pi_{00})\tilde{J}_0}, \text{ for } j \in \tilde{\mathcal{J}_1} \nonumber
\end{align}

Note that we assume that there is no overlap between $Z_j, \text{ for } j \in \tilde{\mathcal{J}_0}$ and $J_1$, which makes our uncontaminated estimate conservative.}

% # how many false negatives?
% if not np.sum(peaks['active'])==len(peaks):
%     nonactid = np.where(peaks['active']==0)[0]
%     pi1 = effectsize.Pi1()
%     pi1.pvalues = peaks['pvals'][nonactid]
%     pi1.estimate(starts=10)
%     NoFalseNegatives = pi1.pi1*len(nonactid)


%From the set up, one would assume you would set pi_0^A = 0.05 = alpha_FDR... are you instead saying you run P&M again?  Can you justify that further and explain why you didn't just assume 5%?
