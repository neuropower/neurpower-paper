{\color{Cyan} Mounting evidence over the last few years suggest that published neuroscience research suffer from low power due to the use of small sample sizes.  Larger sample sizes increase both the chance of detecting a true effect and the chance that a significant result indicates a true effect. } As such, a (prospective) power analysis is a critical component of any paper.  In this work we present a simple way to characterize the spatial signal in a fMRI study with just two parameters, and a direct way to estimate these two parameters based on an existing study. Specifically, using just (1) the proportion of the brain activated and (2) the average effect size in activated brain regions, we can produce closed form power calculations for given sample size, brain volume and smoothness. This procedure allows one to minimize the cost of an fMRI experiment, while preserving a predefined level of statistical power.
The method is evaluated and illustrated using simulations and real neuroimaging data from the Human Connectome Project.  The procedures presented in this paper are made publicly available in a cloud-based toolbox available at www.neuropowertools.org.\\

Keywords: power, fMRI, neuroimaging, sample size, effect size, statistical power
